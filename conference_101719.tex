\documentclass[conference]{IEEEtran}
\IEEEoverridecommandlockouts
% The preceding line is only needed to identify funding in the first footnote.
\usepackage{cite}
\usepackage{amsmath,amssymb,amsfonts}
\usepackage{algorithmic}
\usepackage{graphicx}
\usepackage{textcomp}
\usepackage{xcolor}
\def\BibTeX{{\rm B\kern-.05em{\sc i\kern-.025em b}\kern-.08em
    T\kern-.1667em\lower.7ex\hbox{E}\kern-.125emX}}
\begin{document}

\title{Trends and Technologies in Environmental Monitoring: A Review of Sensors, Communication, and AI-Enabled Systems
%Sensors for Environmental Monitoring: Trends and Challenges
%Sensors for Environmental Monitoring: A Review of Sensors, Communication Technologies, and Applications
\thanks{This work was supported by the National Council for Scientific and Technological Development (CNPq).}
}

\author{\IEEEauthorblockN{Klaus Dieter Kupper}
\IEEEauthorblockA{\textit{Programa de Pós-Graduação}\\ \textit{ em Computação (PPGC)} \\
\textit{University of Vale do Itajaí (UNIVALI)}\\
Itajaí, Brazil \\
klausdk1999@gmail.com}
\and
\IEEEauthorblockN{Jordan Passinato Sausen}
\IEEEauthorblockA{\textit{Programa de Pós-Graduação}\\ \textit{ em Computação (PPGC)} \\
\textit{University of Vale do Itajaí (UNIVALI)}\\
Itajaí, Brazil \\
jordan@univali.br}
\and
\IEEEauthorblockN{Mauricio de Campos}
\IEEEauthorblockA{\textit{Programa de Pós-Graduação}\\ \textit{ em Computação (PPGC)} \\
\textit{University of Vale do Itajaí (UNIVALI)}\\
Itajaí, Brazil \\
mauricio.campos@univali.br}
% \and
% \IEEEauthorblockN{4\textsuperscript{th} Given Name Surname}
% \IEEEauthorblockA{\textit{dept. name of organization (of Aff.)} \\
% \textit{name of organization (of Aff.)}\\
% City, Country \\
% email address or ORCID}
% \and
% \IEEEauthorblockN{5\textsuperscript{th} Given Name Surname}
% \IEEEauthorblockA{\textit{dept. name of organization (of Aff.)} \\
% \textit{name of organization (of Aff.)}\\
% City, Country \\
% email address or ORCID}
% \and
% \IEEEauthorblockN{6\textsuperscript{th} Given Name Surname}
% \IEEEauthorblockA{\textit{dept. name of organization (of Aff.)} \\
% \textit{name of organization (of Aff.)}\\
% City, Country \\
% email address or ORCID}
}

\maketitle

\begin{abstract}
Wireless Sensor Networks (WSNs) are foundational for addressing modern environmental monitoring challenges driven by climate change. This review provides an integrated
analysis of the trends and challenges, examining sensing technologies for water, soil, and air, alongside communication protocols and best practices. We consolidate advances across sensors, networking, and system-level challenges, including energy efficiency, security, and the integration of the Artificial Intelligence of Things (AIoT). By bridging these multidisciplinary domains, this work serves as a foundational guide for future research and the development of next-generation monitoring systems.
\end{abstract}

\begin{IEEEkeywords}
% Internet of Things (IoT), Environment, Sensors, Wireless Sensor Networks (WSN), Soil, Water L, Artificial Intelligence of Things (AIoT), Environmental Monitoring.  
Environmental monitoring, Wireless Sensor Networks, Internet of Things (IoT), LoRaWAN, Artificial Intelligence of Things (AIoT).
\end{IEEEkeywords}

\section{Introduction}
%borga_2014_hydrogeomorphic, bragana_2024_anlise lin_2020_semantic, lo_2015_visual,
The increasing urbanization and climate change have diverse impacts on different layers of society, threatening individuals in vulnerable situations during disasters such as floods, or affecting agricultural production due to climatic variations \cite{jonkman_2005_global}. These phenomena highlight the need for monitoring systems that can provide more data on environmental conditions and help us monitor, analyze, and predict such events \cite{hall_2014_understanding}.

% pule_2017_wireless
When we talk about environmental monitoring, we refer to a wide range of applications and devices. Particularly in remote and hard-to-reach areas, this represents a significant technical challenge. The vastness of these territories, combined with adverse environmental conditions and the growing demand for real-time data, requires technological solutions that are robust, cost-effective, and scalable \cite{chen_2013_natural, yellampalli_2021_wireless }.

% teng_2014_soil , , mohindru_2023_development  \cite{yin_2021_smart, queiroz_2020_sensors} \cite{nr_2025_ai, wu_2023_a}

While many reviews focus on specific aspects such as soil sensors or water level measurement, few provide an integrated perspective combining sensing technologies, communication infrastructures, energy efficiency, artificial intelligence, and security. This review addresses that gap by consolidating advances across soil, water, and air monitoring domains to highlight overlooked technologies and guide future research toward scalable, intelligent, and resilient environmental monitoring solutions.

\section{Sensor Technologies} \label{cap:sensors}

\subsection{Water Level Sensors for Open Environments} \label{subsec:water_level_sensors}

\subsubsection{Hydrostatic Pressure Sensors}
Traditional water level monitoring methods, like staff gauges and limnigraphs, remain common in natural environments. Limnigraph sensors operate on hydrostatic pressure principles, where pressure transducers measure the water column height using the relationship $P = \rho gh$, where $\rho$ is water density, $g$ is gravitational acceleration, and $h$ is height. Their reliability drops in flood conditions due to damage or obstruction risks \cite{santana_2024_development}.

\subsubsection{Ultrasonic Level Sensors}
Ultrasonic sensors estimate distance via acoustic time-of-flight: a short burst (typically 4--10 cycles at 20--200 kHz) is emitted and the echo delay $t$ is converted to distance using $d = \frac{vt}{2}$.

Recent work focused on open-channel flood monitoring demonstrates that low-cost modules can achieve sub-3\% mean error after targeted compensation. MasoudiMoghaddam \textit{et al.}. \cite{mohammadrezamasoudimoghaddam_2024_a} validated a GY-Us42-based design in laboratory and turbulent field conditions (including foamy, aerated surfaces) using: temperature-based sound speed correction, filtering of spurious multipath or splash-induced echoes (median/Hampel), and mechanical alignment to minimize sidewall reflections. Pereira \textit{et al.}. \cite{pereira_2022_evaluation} identified the HC-SR04 as a technically and economically viable option for short spans ($<5$ m) when mounting geometry and obstruction shielding are controlled. Bresnahan \textit{et al.}. \cite{bresnahan_2023_a} emphasized accessibility of the technology, highlighting adoption in education, citizen science, and early-stage research due to low unit cost (US\$5--15) and simple interfacing.

\subsubsection{LiDAR-Based Level Measurement}
LiDAR (Light Detection and Ranging) systems for hydrometric applications derive distance by measuring the propagation characteristics of emitted laser radiation and its backscattered return from a target surface. Two dominant time-of-flight (TOF) architectures are employed in modern (often solid-state) implementations: (i) pulsed TOF and (ii) amplitude modulated continuous wave (AMCW) TOF \cite{li_2022_a}. In pulsed TOF systems, short optical pulses (typically nanosecond to sub-nanosecond width at 905 nm or 1550 nm) are emitted and the round-trip delay \(\Delta t\) is measured with high-resolution time-to-digital converters (TDCs). The distance is obtained by
\begin{equation}
    d = \frac{c\,\Delta t}{2}
\end{equation}
where \(c\) is the speed of light. AMCW TOF LiDAR instead launches a continuously emitted, amplitude-modulated optical carrier; distance is inferred from the measured phase shift \(\Delta \phi\) between transmitted and received envelopes:
\begin{equation}
    d = \frac{c\,\Delta \phi}{4\pi f}
\end{equation}
where \(f\) is the modulation frequency. Pulsed TOF excels at long range and strong ambient light resilience, while AMCW can achieve fine precision at shorter to medium distances with simpler per-pixel electronics \cite{li_2022_a}. Advances in solid-state beam steering (MEMS mirrors, optical phased arrays, rotating polygon replacements) reduce mechanical complexity, enabling compact low-power modules suitable for distributed river monitoring nodes. Key error sources for water level applications include surface specularity (leading to low diffuse return), beam incidence angle, internal temperature drift of laser driver and detector gain, and atmospheric scattering under fog or heavy rain.

Field-oriented studies highlight the performance of low-cost near-infrared LiDAR for hydrometry. An inclined-beam configuration demonstrated by Tamari and Guerrero-Meza leverages suspended sediment and turbidity to enhance backscattered signal strength; higher particle concentrations improved range stability, enabling accurate flash-flood stage retrieval with ±0.08 m error while keeping sensor hardware safely on the river bank \cite{tamari_2016_flash}. Paul \textit{et al.} provided a technical evaluation showing that despite the intrinsically low reflectivity of calm water, a commercial low-cost unit attained relative errors of ~0.1\% over 30–35 m ranges, outperforming typical ultrasonic sensors in maximum span; however, pronounced sensitivity to internal temperature underscored the necessity of thermal compensation (e.g., on-board thermistors with calibration lookup tables or real-time drift modeling) \cite{paul_2020_a}. Complementing these, Santana \textit{et al.} compared a low-cost LiDAR directly against a conventional limnigraph (hydrostatic pressure) installation, reporting equivalence within ±0.05 m to a physical metric scale and negligible degradation from sediment presence, reinforcing robustness for tropical river conditions \cite{santana_2024_development}.

From a systems engineering perspective, LiDAR offers non-contact operation, electromagnetic interference immunity, high precision (typical repeatability $<$0.1\% full scale), and enhanced safety for flood-prone or debris-laden channels. Trade-offs include higher capital cost , susceptibility to adverse meteorological factors (dense fog, heavy rain), and thermal drift requiring compensation. Comparative surveys of liquid level technologies position optical (LiDAR and related) approaches as superior in accuracy and maintenance profile relative to many conventional contact methods, while acknowledging cost and environmental vulnerability constraints \cite{singh_2018_review}.

\subsubsection{Satellite-Based Remote Sensing}

Another method for monitoring water levels in open environments is the use of satellite-based remote sensing. The work \cite{jiang_2024_monitoring} employed a multi-source remote sensing data fusion method via Google Earth Engine to estimate surface water area and water level changes in nine plateau lakes in Yunnan, China, between 2003 and 2022. Their results showed high consistency with in-situ measurements, with deviations under 0.3 meters, and highlighted the influence of climate variability and human activities on lake dynamics.

Similarly, \cite{ali_2024_satellite} used Sentinel-3A radar altimetry to assess the impact of dam operations on water levels in the Mekong River basin, validating satellite-derived levels with in-situ telemetry data and achieving a correlation of 0.98.

Advantages of this method include wide area coverage, long-term consistency, and accessibility to remote regions. Disadvantages include limited spatial resolution, high initial infrastructure costs, and dependency on satellite availability.

\subsection{Water Level Sensors for Controlled Environments}

While the same sensors applied in rivers and lakes may still be suitable, simpler methods like resistive and capacitive sensors are commonly employed for monitoring water levels in tanks and cisterns. Resistive sensors operate by measuring electrical resistance changes as water level varies, typically using conductive materials with power consumption in the micro-watt range.

Surface acoustic wave (SAW) sensors represent emerging passive technologies that can detect water level changes by measuring strain or pressure variations on tank walls, converting them into frequency shifts or response signals \cite{ali_2020_saw, sreejith_2024_modeling}. These sensors operate without active power supply, harvesting energy from the interrogating RF signal, making them ideal for IoT and Industry 5.0 applications requiring minimal maintenance.

Another promising approach involves optical fiber-based sensors. Optical fiber-based sensors rely on the hydrostatic principle of Archimedes, where a floating element alters light transmission through a fiber to reflect changes in water level \cite{ramos_2025_high}. While highly accurate (±1mm resolution), these systems typically have limited measurement ranges and involve more complex calibration procedures, requiring specialized optical components and signal processing units.

\section{Sensor Systems for Detecting Chemicals and Pollutants}

\subsection{Water Quality Sensors}

Recent advancements in sensor technologies have significantly enhanced water quality monitoring. The work by \cite{ferreira_2023_conception} presents the development of a wireless sensor network (WSN) node capable of detecting pH, turbidity, and electrical conductivity (EC) sensors and classifying pollutants in aquatic environments using embedded systems, IoT, and machine learning. The system demonstrated promising results in controlled tests with seawater samples containing oil derivatives, offering low power consumption and effective pollutant classification through distributed processing and embedded neural networks. Complementing this work, \cite{nr_2025_ai} review the application of artificial intelligence (AI) in water quality assessment.

\subsection{Alternative Technologies for Soil Monitoring}

Batteryless sensing technologies have emerged as promising alternatives for soil sensing in remote or hard to access locations. A NFC-based soil sensor that harvests energy from the NFC reader’s magnetic field was introduced by \cite{boada_2018_batteryless}, enabling measurements of temperature, humidity, and soil moisture without relying on batteries. The system integrates a microcontroller and stores data in NDEF format for later retrieval. 

For continuous, high-resolution monitoring applications, Active Heating Optical Frequency Domain Reflectometry (AH-OFDR) provides distributed sensing capabilities along fiber optic cables \cite{sun_2024_highresolution}. This technology combines heated fiber optic cables with OFDR systems, leveraging the principle that soil thermal conductivity varies with moisture content. The system operates by actively heating sections of optical fiber while monitoring temperature changes. Tests validated the method against traditional distributed temperature sensing (DTS) and infrared thermography \cite{sun_2024_highresolution}.

\subsection{Chemical Element Detection and Gas Sensing}

The study by \cite{devkota_2017_saw} investigates SAW sensors for passive detection of gases and chemical vapors in the air, based on the interaction of the compounds with the antenna, which alters the received acoustic signal. SAW sensors are shown to be viable for detecting inorganic gases such as NH$_3$, NO$_2$, SO$_2$, CO, organic vapors such as toluene, ethanol, acetone, and even chemical warfare agents.

The study highlights the importance of material stability in extreme environments, the use of antennas for wireless operation, and low energy consumption, pointing to future research directions in sensitive materials, flexible sensors, and multi-element arrays for simultaneous gas monitoring. Advantages include passive wireless operation, low power consumption (no power required for sensing), and rapid response times (seconds). Disadvantages include temperature sensitivity, selectivity limitations, and drift over long-term operation due to substrate saturation.

\subsection{Air Pollution and Quality Monitoring}

The work by \cite{karagulian_2019_review} analyzes the performance of low-cost sensors (LCS) for monitoring various air pollutants, including carbon monoxide (CO), nitrogen oxides (NO and NO$_2$), ozone (O$_3$), and particulate matter (PM$_{2.5}$). The review highlights the potential of these sensors to expand spatial coverage in both urban and remote areas and assesses different calibration methods such as MLR, ANN, SVR, and RF, considering factors like relative humidity, which significantly affects particulate measurement.

The study by \cite{yi_2015_a} presents a review of air pollution monitoring systems based on wireless sensor networks (WSNs), classifying them into three main categories: Static Sensor Networks (SSN), Community Sensor Networks (CSN), and Vehicular Sensor Networks (VSN), based on the types of sensor carriers. The analysis shows that many current solutions are already viable in terms of spatiotemporal resolution, cost, energy efficiency, ease of deployment, maintenance, and public data accessibility. However, challenges remain, such as the lack of 3D data acquisition, limitations in active monitoring capability, and the use of uncontrolled or semi-controlled carrier, factors that should be improved in the next generations of air pollution monitoring systems.

\section{Challenges and Trends} \label{cap:challenges}

\subsection{Energy-Efficient Communication and System Architecture}

The maturation of wireless sensor networks for environmental monitoring has shifted design priorities toward long range, low maintenance, and multi‑year autonomy \cite{pule_2017_wireless}. Low Power Wide Area Network (LPWAN) paradigms, like Sigfox, LoRa/LoRaWAN, and NB‑IoT, compete to service hydrological, soil, and air-quality deployments by offering kilometer-scale coverage at low bitrates \cite{mekki_2019_a}. Comparative analyses show strategic differentiation: Sigfox leverages narrowband frames and strict duty cycle limits to minimize energy per uplink for tiny, infrequent payloads, but imposes downlink and payload size constraints. LoRaWAN provides adaptive data rates and multiple device classes: Class A (uplink‑initiated, lowest energy), optional Class B (scheduled receive slots), and Class C (near‑continuous listening for low bidirectional latency) at elevated power cost. NB‑IoT, integrated with cellular infrastructure, offers improved QoS, managed mobility, and lower latency, yet synchronization overhead, attach procedures, and OFDM/FDMA waveforms drive higher peak and average currents, shortening lifetime versus Sigfox or Class A LoRa for identical reporting intervals \cite{mekki_2019_a}.

Achieving energy neutrality (average harvested power $\geq$ average consumption) under variable illumination and seasonal conditions is central to scaling dense monitoring networks \cite{shaikh_2016_energy}. Energy-efficient operation across these LPWANs tends to maximize sleep ratios: end devices remain in deep sleep the majority of time, waking briefly for sensing, local preprocessing, and transmission. Adaptive sampling and edge feature extraction lower message frequency by transmitting only aggregated statistics, anomalies, or classifications instead of raw series. Architectural critiques emphasize that while such reductions improve battery life, insufficient end‑to‑end encryption and authentication in cloud-forwarding pipelines expose environmental telemetry to spoofing or replay \cite{pule_2017_wireless}.

\subsection{Large Area Monitoring With Vehicles}

Although LoRaWAN and traditional WSNs are common in large agricultural fields, alternative architectures have emerged to overcome infrastructure, energy, and scalability challenges \cite{yellampalli_2021_wireless}. Deng et al. \cite{deng_2020_novel} developed a hybrid system using passive RFID soil sensors and a patrol vehicle for mobile data collection. This approach offers low cost and energy efficiency by leveraging backscatter communication to send environmental data to a central platform.

Caruso et al. \cite{caruso_2021_drone} introduced a drone-based system where UAVs with LoRa radios periodically collect data from ground sensors. Ideal for areas without fixed infrastructure or where real-time updates aren't essential, the method includes an adaptable model for optimizing flight paths and sensor layout, supporting various drone types and communication protocols.

\subsection{Beat Sensors}

Beat sensors represent an innovative approach in the field of IoT, designed specifically to minimize power consumption and cost. Introduced in 2017, these sensors operate by transmitting only ID codes wirelessly; the receiving unit (RX) decodes sensor data based on the time intervals between transmissions, referred to as “beats.” Applications include power monitoring systems such as Power Beat and DC Current Beat sensors, aimed at energy reduction in residential and commercial buildings \cite{ishibashi_2017_beat}.

Subsequent studies have demonstrated the long-term efficiency of beat sensors in energy-critical IoT scenarios. A 2019 evaluation showed that a Beat sensor powered by a CR2032 coin battery could transmit over 3.8 million ID signals, translating to more than seven years of operation, significantly outperforming conventional IoT sensors with intermittent duty cycles \cite{ishibashi_2019_long}. More recently, a 2025 study applied the beat sensing concept to water-level monitoring, proposing a solar-powered system using a \$ 72.5 BoM sensor node with LoRa communication and a quasi-isotropic $\Omega$-antenna. The device operated continuously for 21 days without a battery, achieving a 2 km communication range and consuming less than 100 $\mu$W \cite{dao_2025_lowcost}.

\subsection{Artificial Intelligence with Internet of Things (AIoT)}

The integration of Artificial Intelligence into IoT systems, known as AIoT, enables smarter, autonomous applications in areas like smart cities, industry, and environmental monitoring. As described by \cite{ghosh_2018_artificial}, this marks a shift toward cyber-physical systems that combine embedded devices, human interaction, and real-time analytics. While AI enhances predictive and adaptive capabilities, it also introduces challenges such as data security and system complexity.

AI-enabled sensors are becoming more context-aware and energy-efficient, supporting applications like predictive maintenance and smart routing \cite{mukhopadhyay_2021_artificial}. In environmental monitoring, \cite{ferreira_2023_conception} developed a sensor node using embedded neural networks for pollutant classification, while \cite{nr_2025_ai} reviewed AI's role in water quality analysis, noting its potential benefits and data-related limitations.

\subsection{Security and Privacy Challenges}

The widespread integration of embedded systems in IoT architectures introduces critical security challenges, particularly as these systems perform real-time sensing and decision-making in sensitive contexts. Their constrained nature and the tendency to treat security as secondary in design make them vulnerable to attacks \cite{pimentel_2017_exploring}. As embedded systems now support safety-critical applications, ensuring reliability and resilience becomes essential \cite{koulamas_2018_realtime}. A compromised device can threaten the entire IoT network, highlighting the need for robust, integrated security, especially in edge processing, where minimizing data transfer also reduces the attack surface \cite{tien_2017_internet}.

In addition to software risks, hardware-level threats such as microarchitectural attacks are gaining attention. These stealthy exploits bypass traditional protections and compromise trust mechanisms \cite{fournaris_2017_exploiting}. Middleware connecting heterogeneous systems is particularly exposed if hardware is not adequately secured. Manufacturers must adopt secure-by-design principles and deliver timely patches, while architects must understand the full attack taxonomy to implement end-to-end protections \cite{fournaris_2017_exploiting, pimentel_2017_exploring}.

\section{Conclusion and Future Perspectives} \label{cap:conclusion}

Environmental monitoring technologies are evolving rapidly to meet the challenges posed by climate change, urbanization, and resource management. This review presented the current landscape and emerging directions in sensor development, ranging from conventional water level and air pollution sensors to more advanced, energy-efficient, and batteryless alternatives.

The review also helps identify the most suitable sensing technologies based on context. For water level monitoring, ultrasonic sensors are ideal for short ranges (under 5 meters), while LiDAR offers better performance for longer distances. Satellite-based remote sensing is well-suited for large areas or hard-to-access regions. Soil monitoring benefits from technologies like NFC-powered sensors and optical fiber, while large-scale deployments can rely on mobile data collection using vehicles or UAVs.

In air quality monitoring, and to a certain extent in other domains, sensor affordability and maturity are no longer the main limitations. The current challenges lie in expanding data acquisition, maximizing the value of collected data, and securing communication in increasingly distributed sensor networks.

There is significant potential in combining the innovations discussed throughout this review. For example, AIoT can improve pollutant classification, Beat sensors enable long-term monitoring in remote areas, and applying security by design is critical as networks scale. Notably, Artificial Intelligence can be a powerful tool for prediction, anomaly detection, and autonomous response, but its effectiveness depends on the availability of high-quality, large-scale data. This creates a natural synergy with the broader push for denser and more distributed sensor deployments in smart cities, agriculture, and environmental systems.

As sensor networks scale in size and complexity, ensuring data security and integrity becomes increasingly critical. Security threats can undermine not only data accuracy but also the reliability of decision-making systems in critical scenarios, potentially triggering false alerts or skewing AI-driven analyses. Security must therefore be treated as a foundational component of system design, not an afterthought.

% Environmental monitoring technologies are evolving rapidly to meet the challenges posed by climate change, urbanization, and resource management within the context of Industry 5.0. This comprehensive review presented the current landscape and emerging directions in sensor development, encompassing detailed analysis of operating principles, manufacturing technologies, and power management strategies for next-generation monitoring systems.

% The integration of Power Electronics and Control systems with environmental sensing represents a paradigm shift toward sustainable and energy-efficient monitoring solutions. Low-power design strategies utilizing ARM Cortex-M series microcontrollers achieve power consumption below 100 $\mu$W, while energy harvesting technologies enable perpetual operation without battery replacement. These advances are particularly relevant for Industry 5.0 applications requiring minimal maintenance and environmental impact.

% This review identifies optimal sensing technologies based on application context and power requirements. For water level monitoring, hydrostatic pressure sensors provide highest accuracy (±0.1\% full scale) for controlled environments, while ultrasonic sensors offer the best compromise between cost (\$50-200) and performance for short ranges. LiDAR systems excel in challenging environments with superior accuracy (0.1\% error) but require higher power budgets (2-5W). Satellite-based remote sensing enables large-scale monitoring with minimal ground infrastructure.

% Advanced soil monitoring technologies demonstrate the potential of batteryless sensing, with NFC-based sensors operating from harvested RF energy and optical fiber systems providing distributed sensing capabilities. The integration of ARM-based microcontrollers and specialized sensor ICs enables sophisticated signal processing and calibration directly at sensor nodes.

% The convergence of multiple technological domains creates significant opportunities for next-generation environmental monitoring systems. AIoT implementations utilizing ARM Cortex-M55 processors with integrated AI acceleration enable real-time pollutant classification and predictive analytics directly at sensor nodes, reducing communication overhead and improving response times. Beat sensors demonstrate the potential for ultra-low-power operation, achieving seven-year battery life from coin cells while maintaining 2 km communication ranges through optimized LoRa protocols.

% Chemical sensing technologies benefit from advances in semiconductor manufacturing, with SAW sensors achieving ppb-level detection through precise lithographic fabrication of interdigital transducers, while MOS sensors integrate temperature control and signal processing within single-chip solutions. The combination of these sensing modalities with energy harvesting technologies enables deployment in previously inaccessible locations.

% Security considerations become paramount as sensor networks integrate with Industry 5.0 infrastructure. Hardware-level security implementations utilizing ARM TrustZone technology and secure boot mechanisms provide protection against microarchitectural attacks, while encrypted communication protocols ensure data integrity across distributed sensor networks. The adoption of secure-by-design principles is essential for maintaining system reliability in safety-critical applications.

% Future research directions should focus on the integration of emerging technologies including quantum sensors for ultra-high sensitivity applications, neuromorphic computing for edge AI processing, and advanced materials such as graphene-based sensors for improved selectivity and stability. The development of standardized interfaces and protocols will be crucial for enabling interoperability across heterogeneous sensor networks.

% By consolidating advances in sensor physics, embedded systems, power electronics, and cybersecurity, this review provides a comprehensive foundation for developing next-generation environmental monitoring systems. These integrated approaches enable more scalable, autonomous, and resilient solutions capable of supporting climate adaptation, resource management, and sustainable development goals within the Industry 5.0 framework.

\begin{thebibliography}{00}
% 1
\bibitem{jonkman_2005_global} S. N. Jonkman, 
``Global Perspectives on Loss of Human Life Caused by Floods,'' 
Natural Hazards, vol. 34, pp. 151--175, Feb. 2005. doi: 10.1007/s11069-004-8891-3.

\bibitem{hall_2014_understanding} J. Hall \textit{et al.}, 
``Understanding Flood Regime Changes in Europe: A State-of-the-Art Assessment,'' 
Hydrology and Earth System Sciences, vol. 18, pp. 2735--2772, Jul. 2014. doi: 10.5194/hess-18-2735-2014.

\bibitem{chen_2013_natural} D. Chen, Z. Liu, L. Wang, M. Dou, J. Chen, and H. Li, 
``Natural Disaster Monitoring with Wireless Sensor Networks: A Case Study of Data-Intensive Applications upon Low-Cost Scalable Systems,'' 
Mobile Networks and Applications, vol. 18, pp. 651--663, Aug. 2013. doi: 10.1007/s11036-013-0456-9.

\bibitem{yellampalli_2021_wireless} S. Yellampalli, 
*Wireless Sensor Networks - Design, Deployment and Applications*, IntechOpen, Sep. 2021. doi: 10.5772/intechopen.77917. [Online]. Available: https://www.intechopen.com/books/8086
% 5
\bibitem{santana_2024_development} V. Santana, R. E. Salustiano, and R. Tiezzi, ``Development and Calibration of a Low-Cost LIDAR Sensor for Water Level Measurements,'' \emph{Flow Measurement and Instrumentation}, vol. 2024, pp. 102729--102729, Oct. 2024. doi: 10.1016/j.flowmeasinst.2024.102729.

\bibitem{mohammadrezamasoudimoghaddam_2024_a} M. MasoudiMoghaddam, J. Yazdi, and M. Shahsavandi, 
``A Low-Cost Ultrasonic Sensor for Online Monitoring of Water Levels in Rivers and Channels,'' 
Flow Measurement and Instrumentation, vol. 102, p. 102777, Dec. 2024. Elsevier BV. doi: 10.1016/j.flowmeasinst.2024.102777.

\bibitem{pereira_2022_evaluation} T. S. R. Pereira, T. P. de Carvalho, T. A. Mendes, and K. T. M. Formiga, 
``Evaluation of Water Level in Flowing Channels Using Ultrasonic Sensors,'' 
Sustainability, vol. 14, p. 5512, May 2022. doi: 10.3390/su14095512.

\bibitem{bresnahan_2023_a} P. Bresnahan, E. Briggs, B. Davis, A. Rodriguez, L. Edwards, C. Peach, N. Renner, H. Helling, and M. Merrifield, 
``A Low-Cost, DIY Ultrasonic Water Level Sensor for Education, Citizen Science, and Research,'' 
Oceanography, vol. 36, 2023. doi: 10.5670/oceanog.2023.101.

\bibitem{li_2022_a} N. Li, C. P. Ho, J. Xue, L. W. Lim, G. Chen, Y. H. Fu, and L. Y. T. Lee, 
``A Progress Review on Solid-State LiDAR and Nanophotonics-Based LiDAR Sensors,'' 
Laser and Photonics Reviews, vol. 16, p. 2100511, Aug. 2022. doi: 10.1002/lpor.202100511.

\bibitem{paul_2020_a} J. D. Paul, W. Buytaert, and N. Sah, 
``A Technical Evaluation of Lidar-Based Measurement of River Water Levels,'' 
Water Resources Research, vol. 56, Apr. 2020. doi: 10.1029/2019wr026810.
% 10
\bibitem{tamari_2016_flash} S. Tamari and V. Guerrero-Meza, 
``Flash Flood Monitoring with an Inclined Lidar Installed at a River Bank: Proof of Concept,'' 
Remote Sensing, vol. 8, p. 834, Oct. 2016. doi: 10.3390/rs8100834.

\bibitem{jiang_2024_monitoring} Z. Jiang and L. Hong, 
``Monitoring of Surface Water Area and Water Level Changes in Nine Plateau Lakes in Yunnan and Analysis of Influencing Factors,'' 
in *Proc. 2024 IEEE 6th Advanced Information Management, Communicates, Electronic and Automation Control Conf. (IMCEC)*, 
pp. 1027--1031, May 2024. doi: 10.1109/imcec59810.2024.10575197.

\bibitem{ali_2024_satellite} T. Ali, A. Zaidi, J. Rehman, F. Noor, F. Naz, and S. Jamali, 
``Satellite Radar Altimetry Insights into Dam-Induced Changes and Accuracy of Water Level Estimation for the Mekong River,'' 
in *Proc. IGARSS 2022 - IEEE Int. Geosci. Remote Sens. Symp.*, vol. 570, pp. 5063--5066, Jul. 2024. 
doi: 10.1109/igarss53475.2024.10642418.

\bibitem{ali_2020_saw} S. F. Ali, N. Mandal, P. Maurya, and A. Lata, 
``SAW Sensor Based a Novel Hydrostatic Liquid Level Measurement,'' 
in *Proc. IECON 2020 - 46th Annual Conf. IEEE Industrial Electronics Society*, pp. 724--729, Oct. 2020. 
doi: 10.1109/iecon43393.2020.9254540.

\bibitem{sreejith_2024_modeling} V. S. Sreejith and H. Zhang, 
``Modeling and Testing of a Highly Sensitive Surface Acoustic Wave Pressure Sensor for Liquid Depth Measurements,'' 
Sensors and Actuators A: Physical, vol. 372, p. 115377, Jul. 2024. doi: 10.1016/j.sna.2024.115377.
% 15
\bibitem{ramos_2025_high} C. C. Ramos, J. Preizal, X. Hu, C. Caucheteur, G. Woyessa, O. Bang, A. M. Rocha, and R. Oliveira, 
``High Resolution Liquid Level Sensor Based on Archimedes’ Law of Buoyancy Using Polymer Optical Fiber Bragg Gratings,'' 
Measurement, vol. 252, p. 117368, Mar. 2025. Elsevier BV. doi: 10.1016/j.measurement.2025.117368.

\bibitem{pule_2017_wireless} M. Pule, A. Yahya, and J. Chuma, 
``Wireless Sensor Networks: A Survey on Monitoring Water Quality,'' 
Journal of Applied Research and Technology, vol. 15, pp. 562--570, Dec. 2017. doi: 10.1016/j.jart.2017.07.004.

\bibitem{mekki_2019_a}
K. Mekki, E. Bajic, F. Chaxel, and F. Meyer, "A Comparative Study of LPWAN Technologies for Large-Scale IoT Deployment," \emph{ICT Express}, vol. 5, pp. 1--7, Mar. 2019. doi: 10.1016/j.icte.2017.12.005.

\bibitem{ferreira_2023_conception} Y. Ferreira, C. Silvério, and J. Viana, 
``Conception and Design of WSN Sensor Nodes Based on Machine Learning, Embedded Systems and IoT Approaches for Pollutant Detection in Aquatic Environments,'' 
IEEE Access, vol. 11, pp. 117040--117052, Jan. 2023. doi: 10.1109/access.2023.3325760.

\bibitem{nr_2025_ai} W. B. N.R, S. Palarimath, H. Gunasekaran, S. W. Haidar, S. G. S. Subitha, and J. Sarmila, 
``AI Modeling and Water Quality Sensing Technique Proffers Water Security: An Open Review,'' 
in *Proc. 2022 Int. Conf. Electronics and Renewable Systems (ICEARS)*, pp. 1028--1034, Feb. 2025. doi: 10.1109/icears64219.2025.10940077.

\bibitem{boada_2018_batteryless} M. Boada, A. Lazaro, R. Villarino, and D. Girbau, 
``Battery-Less Soil Moisture Measurement System Based on a NFC Device With Energy Harvesting Capability,'' 
IEEE Sensors Journal, vol. 18, pp. 5541--5549, Jul. 2018. 
doi: 10.1109/jsen.2018.2837388.

\bibitem{sun_2024_highresolution} C. Sun, C.-S. Tang, F. Vahedifard, Q. Cheng, A. Dong, T.-F. Gao, and B. Shi, "High-resolution monitoring of soil infiltration using distributed fiber optic," J. Hydrol., vol. 640, p. 131691, Jul. 2024, doi: 10.1016/j.jhydrol.2024.131691.

\bibitem{devkota_2017_saw} J. Devkota, P. Ohodnicki, and D. Greve, 
``SAW Sensors for Chemical Vapors and Gases,'' 
Sensors, vol. 17, p. 801, Apr. 2017. 
doi: 10.3390/s17040801.

\bibitem{karagulian_2019_review} F. Karagulian, M. Barbiere, A. Kotsev, L. Spinelle, M. Gerboles, F. Lagler, N. Redon, S. Crunaire, and A. Borowiak, 
``Review of the Performance of Low-Cost Sensors for Air Quality Monitoring,'' 
Atmosphere, vol. 10, p. 506, Aug. 2019. doi: 10.3390/atmos10090506.

\bibitem{yi_2015_a} W. Yi, K. Lo, T. Mak, K. Leung, Y. Leung, and M. Meng, ``A survey of wireless sensor network based air pollution monitoring systems,'' \emph{Sensors}, vol. 15, pp. 31392--31427, Dec. 2015. doi: 10.3390

\bibitem{deng_2020_novel} F. Deng, P. Zuo, K. Wen, and X. Wu, 
``Novel soil environment monitoring system based on RFID sensor and LoRa,'' 
Computers and Electronics in Agriculture, vol. 169, p. 105169, Feb. 2020. doi: 10.1016/j.compag.2019.105169.

\bibitem{caruso_2021_drone} A. Caruso, S. Chessa, S. Escolar, J. Barba, and J. C. Lopez, “Collection of Data with Drones in Precision Agriculture: Analytical Model and LoRa Case Study,” IEEE Internet of Things Journal, vol. 2021, pp. 1–1, 2021, doi: 10.1109/JIOT.2021.3075561.

\bibitem{ishibashi_2017_beat} 
K. Ishibashi, R. Takitoge, and S. Ishigaki, 
"Beat sensors IoT technology suitable for energy saving," 
in *Proc. 2017 7th Int. Conf. on Integrated Circuits, Design, and Verification (ICDV)*, pp. 52--55, Oct. 2017. doi: 10.1109/icdv.2017.8188637.

\bibitem{ishibashi_2019_long} 
K. Ishibashi, R. Takitoge, D. Manyvone, N. Ono, and S. Yamaguchi, 
"Long battery life IoT sensing by Beat Sensors," 
in *Proc. 2019 IEEE Int. Conf. on Industrial Cyber Physical Systems (ICPS)*, pp. 430--435, May 2019. doi: 10.1109/icphys.2019.8780159.

\bibitem{dao_2025_lowcost} 
M.-H. Dao, K. Ishibashi, T.-A. Nguyen, D.-H. Bui, H. Hirayma, T.-A. Tran, and X.-T. Tran, 
"Low-cost, high accuracy, and long communication range energy-harvesting Beat Sensor with LoRa and $\Omega$-antenna for water-level monitoring,"
*IEEE Sensors Journal*, vol. 25, pp. 1--1, Jan. 2025. doi: 10.1109/jsen.2025.3533014.

\bibitem{ghosh_2018_artificial} A. Ghosh, D. Chakraborty, and A. Law, 
'Artificial Intelligence in Internet of Things,' 
CAAI Transactions on Intelligence Technology, vol. 3, pp. 208--218, Dec. 2018. doi: 10.1049/trit.2018.1008. [Online]. Available: https://ietresearch.onlinelibrary.wiley.com/doi/10.1049/trit.2018.1008

\bibitem{mukhopadhyay_2021_artificial} S. C. Mukhopadhyay, S. K. S. Tyagi, N. K. Suryadevara, V. Piuri, F. Scotti, and S. Zeadally, 
``Artificial Intelligence-Based Sensors for Next Generation IoT Applications: A Review,'' 
IEEE Sensors Journal, vol. 21, pp. 1--1, 2021. doi: 10.1109/jsen.2021.3055618.

\bibitem{pimentel_2017_exploring}
Pimentel, A. D. (2017). Exploring exploration: A tutorial introduction to embedded systems design space exploration. \textit{IEEE Design \& Test}, 1, 77--90. doi:10.1109/MDAT.2016.2626445.

\bibitem{koulamas_2018_realtime}
Koulamas, C., \& Lazarescu, M. (2018). Real-time embedded systems: Present and future. Electronics, 9, 205. doi:10.3390/electronics7090205.

\bibitem{tien_2017_internet} J. M. Tien, 
``Internet of Things, Real-Time Decision Making, and Artificial Intelligence,'' 
Annals of Data Science, vol. 4, pp. 149--178, May 2017.

\bibitem{fournaris_2017_exploiting}
Fournaris, A., Pocero Fraile, L., \& Koufopavlou, O. (2017). Exploiting hardware vulnerabilities to attack embedded system devices: A survey of potent microarchitectural attacks. Electronics,3, 52. doi:10.3390/electronics6030052.

\bibitem{singh_2018_review}
P. Singh and R. Kumar, "Liquid Level Measurement Technologies: A Comparative Review of Conventional and Optical Methods," \emph{Measurement and Control}, vol. 51, no. 7-8, pp. 302--313, 2018. doi: 10.1177/0020294018788495.

\bibitem{shaikh_2016_energy}
F. K. Shaikh and S. Zeadally, "Energy Harvesting in Wireless Sensor Networks: A Comprehensive Review," \emph{Renewable and Sustainable Energy Reviews}, vol. 55, pp. 1041--1054, Mar. 2016. doi:10.1016/j.rser.2015.11.010.

\end{thebibliography}



% \bibitem{bragana_2024_anlise} C. G. J. F. N. Bragança, 
% ``Análise da Influência dos Fenômenos El Niño e La Niña na Ocorrência de Eventos Climáticos de Seca e Enchente,'' 
% Dissertação de Mestrado, 2024.% \bibitem{borga_2014_hydrogeomorphic} M. Borga, M. Stoffel, L. Marchi, F. Marra, and M. Jakob, 
% ``Hydrogeomorphic Response to Extreme Rainfall in Headwater Systems: Flash Floods and Debris Flows,'' 
% Journal of Hydrology, vol. 518, pp. 194--205, Oct. 2014. doi: 10.1016/j.jhydrol.2014.05.022.
% \bibitem{bragana_2024_anlise} C. G. J. F. N. Bragança, 
% ``Análise da Influência dos Fenômenos El Niño e La Niña na Ocorrência de Eventos Climáticos de Seca e Enchente,'' 
% Dissertação de Mestrado, 2024.
% \bibitem{javaid_2021_sensors} M. Javaid, A. Haleem, S. Rab, R. P. Singh, and R. Suman, ``Sensors for daily life: A review,'' \emph{Sensors International}, vol. 2, p. 100121, 2021. doi: 10.1016/j.sintl.2021.100121.

% \bibitem{mohindru_2023_development} P. Mohindru, ``Development of liquid level measurement technology: A review,'' \emph{Flow Measurement and Instrumentation}, vol. 89, p. 102295, Mar. 2023. doi: 10.1016/j.flowmeasinst.2022.102295.
% \bibitem{wu_2023_a} Z. Wu, Y. Huang, K. Huang, K. Yan, and H. Chen, ``A Review of Non-Contact Water Level Measurement Based on Computer Vision and Radar Technology,'' \emph{Water}, vol. 15, p. 3233, Jan. 2023. doi: 10.3390/w15183233.
% \bibitem{teng_2014_soil} Y. Teng, J. Wu, S. Lu, Y. Wang, X. Jiao, and L. Song, 
% ``Soil and Soil Environmental Quality Monitoring in China: A Review,'' 
% Environment International, vol. 69, pp. 177--199, Aug. 2014. doi: 10.1016/j.envint.2014.04.014.
% \bibitem{behroozpour_2017_lidar} B. Behroozpour, P. A. M. Sandborn, M. C. Wu, and B. E. Boser, 
% ``Lidar System Architectures and Circuits,'' 
% IEEE Communications Magazine, vol. 55, pp. 135--142, Oct. 2017. doi: 10.1109/mcom.2017.1700030.
% \bibitem{fernandezdiaz_2014_early} J. C. Fernandez-Diaz, C. L. Glennie, W. E. Carter, R. L. Shrestha, M. P. Sartori, A. Singhania, C. J. Legleiter, and B. T. Overstreet, 
% ``Early Results of Simultaneous Terrain and Shallow Water Bathymetry Mapping Using a Single-Wavelength Airborne LiDAR Sensor,'' 
% IEEE Journal of Selected Topics in Applied Earth Observations and Remote Sensing, vol. 7, pp. 623--635, Feb. 2014. 
% doi: 10.1109/jstars.2013.2265255.
% \bibitem{smart_2009_river} G. Smart, J. Bind, and M. Duncan, 
% ``River bathymetry from conventional LiDAR using water surface returns,'' 
% in *Proc. 18th World IMACS / MODSIM Congress*, 2009. [Online]. Available: https://mssanz.org.au/modsim09/F13/smart.pdf
% \bibitem{haroun_2021_progress} A. Haroun, X. Le, S. Gao, B. Dong, T. He, Z. Zhang, F. Wen, S. Xu, and C. Lee, 
% ``Progress in Micro/Nano Sensors and Nanoenergy for Future AIoT-Based Smart Home Applications,'' 
% Nano Express, vol. 2, p. 022005, Apr. 2021. doi: 10.1088/2632-959x/abf3d4.
% \bibitem{akhileshnagpure_2022_water} A. Nagpure, G. Reddy, and S. Naidu, 
% ``Water Level Controller Using Ultrasonic Sensor,'' 
% International Journal for Research in Applied Science and Engineering Technology, vol. 10, pp. 242--257, Dec. 2022. 
% doi: 10.22214/ijraset.2022.47856.
% \bibitem{queiroz_2020_sensors} D. M. de Queiroz, A. L. de F. Coelho, D. S. M. Valente, and J. K. Schueller, 
% ``Sensors Applied to Digital Agriculture: A Review,'' 
% Revista Ciência Agronômica, vol. 51, 2020. doi: 10.5935/1806-6690.20200086.
% \bibitem{lin_2020_semantic} F. Lin, Z. Yu, Q. Jin, and A. You, 
% ``Semantic Segmentation and Scale Recognition–Based Water-Level Monitoring Algorithm,'' 
% Journal of Coastal Research, vol. 105, 2020. doi: 10.2112/jcr-si105-039.1.
% \bibitem{lo_2015_visual} S.-W. Lo, J.-H. Wu, F.-P. Lin, and C.-H. Hsu, 
% ``Visual Sensing for Urban Flood Monitoring,'' 
% Sensors, vol. 15, pp. 20006--20029, Aug. 2015. doi: 10.3390/s150820006.
% \bibitem{pule_2017_wireless} M. Pule, A. Yahya, and J. Chuma, 
% ``Wireless Sensor Networks: A Survey on Monitoring Water Quality,'' 
% Journal of Applied Research and Technology, vol. 15, pp. 562--570, Dec. 2017. doi: 10.1016/j.jart.2017.07.004.
% \bibitem{viscarrarossel_2016_soil} R. A. Viscarra Rossel and J. Bouma, 
% ``Soil sensing: A new paradigm for agriculture,'' 
% Agricultural Systems, vol. 148, pp. 71--74, Oct. 2016. doi: 10.1016/j.agsy.2016.07.001.
% \bibitem{yin_2021_smart} H. Yin, Y. Cao, B. Marelli, X. Zeng, A. J. Mason, and C. Cao, 
% ``Smart Agriculture Systems: Soil Sensors and Plant Wearables for Smart and Precision Agriculture (Adv. Mater. 20/2021),'' 
% Advanced Materials, vol. 33, p. 2170156, May 2021. doi: 10.1002/adma.202170156. 
% \bibitem{yukawa_2025_an} C. Yukawa, T. Oda, T. Sato, M. Hirota, K. Katayama, and L. Barolli, 
% ``An Intelligent Water Level Estimation System Considering Water Level Device Gauge Image Recognition and Wireless Sensor Networks,'' 
%Journal of Sensor and Actuator Networks, vol. 14, p. 13, Jan. 2025. doi: 10.3390/jsan14010013.
% \bibitem{borga_2014_hydrogeomorphic} M. Borga, M. Stoffel, L. Marchi, F. Marra, and M. Jakob, 
% ``Hydrogeomorphic Response to Extreme Rainfall in Headwater Systems: Flash Floods and Debris Flows,'' 
% Journal of Hydrology, vol. 518, pp. 194--205, Oct. 2014. doi: 10.1016/j.jhydrol.2014.05.022.
% \bibitem{iqbal_2021_how} U. Iqbal, P. Perez, W. Li, and J. Barthelemy, 
% ``How Computer Vision Can Facilitate Flood Management: A Systematic Review,'' 
% International Journal of Disaster Risk Reduction, vol. 53, p. 102030, Feb. 2021. doi: 10.1016/j.ijdrr.2020.102030.
\end{document}
