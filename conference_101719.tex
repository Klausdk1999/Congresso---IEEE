\documentclass[conference]{IEEEtran}
\IEEEoverridecommandlockouts
% The preceding line is only needed to identify funding in the first footnote. If that is unneeded, please comment it out.
\usepackage{cite}
\usepackage{amsmath,amssymb,amsfonts}
\usepackage{algorithmic}
\usepackage{graphicx}
\usepackage{textcomp}
\usepackage{xcolor}
\def\BibTeX{{\rm B\kern-.05em{\sc i\kern-.025em b}\kern-.08em
    T\kern-.1667em\lower.7ex\hbox{E}\kern-.125emX}}
\begin{document}

\title{Sensors for Environmental Monitoring: Trends and Challenges
%Sensors for Environmental Monitoring: A Review of Sensors, Communication Technologies, and Applications
\thanks{Identify applicable funding agency here. If none, delete this.}
}

\author{\IEEEauthorblockN{1\textsuperscript{st} Klaus Dieter Kupper}
\IEEEauthorblockA{\textit{dept. name of organization (of Aff.)} \\
\textit{name of organization (of Aff.)}\\
City, Country \\
email address or ORCID}
\and
\IEEEauthorblockN{2\textsuperscript{nd} Jordan}
\IEEEauthorblockA{\textit{dept. name of organization (of Aff.)} \\
\textit{name of organization (of Aff.)}\\
City, Country \\
email address or ORCID}
% \and
% \IEEEauthorblockN{3\textsuperscript{rd} Given Name Surname}
% \IEEEauthorblockA{\textit{dept. name of organization (of Aff.)} \\
% \textit{name of organization (of Aff.)}\\
% City, Country \\
% email address or ORCID}
% \and
% \IEEEauthorblockN{4\textsuperscript{th} Given Name Surname}
% \IEEEauthorblockA{\textit{dept. name of organization (of Aff.)} \\
% \textit{name of organization (of Aff.)}\\
% City, Country \\
% email address or ORCID}
% \and
% \IEEEauthorblockN{5\textsuperscript{th} Given Name Surname}
% \IEEEauthorblockA{\textit{dept. name of organization (of Aff.)} \\
% \textit{name of organization (of Aff.)}\\
% City, Country \\
% email address or ORCID}
% \and
% \IEEEauthorblockN{6\textsuperscript{th} Given Name Surname}
% \IEEEauthorblockA{\textit{dept. name of organization (of Aff.)} \\
% \textit{name of organization (of Aff.)}\\
% City, Country \\
% email address or ORCID}
}

\maketitle

\begin{abstract}
Wireless Sensor Networks (WSNs) are foundational for addressing modern environmental monitoring challenges driven by climate change. This review provides an integrated analysis of the state-of-the-art, examining sensing technologies for water, soil, and air, alongside communication protocols and best practices. We consolidate advances across sensors, networking, and system-level challenges, including energy efficiency, security, and the integration of the Artificial Intelligence of Things (AIoT). By bridging these multidisciplinary domains, this work serves as a foundational guide for future research and the development of next-generation monitoring systems.
\end{abstract}

\begin{IEEEkeywords}
Internet of Things (IoT), Environment, Sensors, Wireless Sensor Networks (WSN), Soil, Water L, Artificial Intelligence of Things (AIoT), Environmental Monitoring.
\end{IEEEkeywords}

\section{Introduction}
The increasing urbanization and climate change have diverse impacts on different layers of society, threatening individuals in vulnerable situations during disasters such as floods, or affecting agricultural production due to climatic variations \cite{jonkman_2005_global, hall_2014_understanding, bragana_2024_anlise, borga_2014_hydrogeomorphic}. These phenomena highlight the need for effective monitoring systems that can provide more data on environmental conditions and help us monitor, analyze, and predict such events \cite{hall_2014_understanding, lin_2020_semantic, lo_2015_visual, iqbal_2021_how}.

When we talk about environmental monitoring, we refer to a wide range of applications and devices. Particularly in remote and hard-to-reach areas, this represents a significant technical challenge. The vastness of these territories, combined with adverse environmental conditions and the growing demand for real-time data, requires technological solutions that are robust, cost-effective, and scalable \cite{chen_2013_natural, yellampalli_2021_wireless, pule_2017_wireless}.

This literature review is structured into five main sections. First, it addresses the sensors and technologies used in environmental monitoring, categorizing them according to their application domains (hydrological, soil, and air quality). Next, the role of Wireless Sensor Networks and associated communication technologies is discussed. Then, the main challenges and current trends in the field of environmental monitoring are presented, focusing on technological innovations. Finally, conclusions and future perspectives are outlined.

Although previous reviews have thoroughly examined sensors for soil monitoring, such as those by \cite{yin_2021_smart, teng_2014_soil, queiroz_2020_sensors}, or water level measurement techniques as in \cite{mohindru_2023_development, nr_2025_ai, wu_2023_a}, there is a notable lack of recent studies that comprehensively integrate both sensor technologies and communication infrastructures, presenting emerging concepts in environmental monitoring and going beyond specific applications.

This review aims to fill that gap by offering a broad perspective that covers sensor technologies applied to the monitoring of soil, water, and/or air/chemical compounds, along with an analysis of wireless communication technologies, energy efficiency strategies, artificial intelligence and security challenges in WSN deployments for environmental monitoring. The main contribution of this work is to consolidate multidisciplinary advances, highlight underexplored technologies, and provide a comparative analysis to guide future research and applications in the development of environmental monitoring projects.

\section{Level and range sensors} \label{cap:sensors}

% Our daily lives are full of sensors constantly collecting data about the environment around us. These sensors can be classified in many different ways. Some relevant categorization examples include: active and passive sensors, where active sensors emit some type of signal to measure a physical quantity, while passive sensors only capture signals that already exist in the environment \cite{javaid_2021_sensors}. Another way to classify sensors is by how they interact with the environment, where contact sensors measure physical quantities directly in contact with the medium, while non-contact sensors measure physical quantities without needing direct contact \cite{javaid_2021_sensors, mohindru_2023_development, wu_2023_a}. Finally, there are many other ways to classify sensors;

% In this work, we chose to group sensors according to the type of environment where they are applied, such as water level sensors, air quality sensors, and soil sensors, in order to better guide research in those environments.

\subsection{Water Level Sensors for Open Environments} \label{subsec:water_level_sensors}

Simple methods for monitoring water levels, such as marked rulers or staff gauges, are still widely used in natural environments like rivers and lakes. Other common techniques range from basic visual tools to floating or contact-based sensors, including pressure sensors and limnigraphs, which measure the hydrostatic pressure of the water column. A limnigraph sensor was evaluated in the comparative study by \cite{santana_2024_development}, showing accurate readings and resistance to parameters like water turbidity. However, the same study highlights a key limitation: reduced reliability in harsh conditions. In flood scenarios with strong currents and high debris loads, these sensors may become damaged or obstructed.

For non-contact monitoring applications, this review highlights three approaches: ultrasonic sensors, LiDAR sensors, and remote monitoring via satellite imagery.

Ultrasonic water level sensors emit sound waves and measure the time it takes for the waves to return to the sensor. These sensors are widely used due to their accuracy and ability to operate in environments with temperature and pressure variations \cite{mohammadrezamasoudimoghaddam_2024_a, pereira_2022_evaluation}. For example, the ultrasonic sensor model GY-Us42 was tested, showing that the average error of the device is less than 3\% \cite{mohammadrezamasoudimoghaddam_2024_a}. Another model, the HC-SR04, was also evaluated as a technically and economically viable option for water level monitoring \cite{pereira_2022_evaluation}, and is a good choice for education, citizen science, and research due to its low cost \cite{bresnahan_2023_a}.

LiDAR sensors use optical waves to measure distances and are widely used in metrology, environmental monitoring, archaeology, and robotics \cite{behroozpour_2017_lidar, li_2022_a}.
These sensors have been explored as a low-cost alternative for measuring water levels from bridges, with lab and field tests showing good accuracy (error around 0.1\%), though subject to variations due to sensor temperature and water surface roughness \cite{paul_2020_a}. LiDAR sensors installed on riverbanks for flood monitoring were also tested with good results, the study showed that suspended particles in the water positively affected reading accuracy and that the sensor could also be used to detect suspended particle concentration \cite{tamari_2016_flash}. Another study compared the TF-mini LiDAR sensor with limnigraph pressure sensors, highlighting the benefits of LiDAR’s non-contact measurement method over the contact-based limnigraph method and validating LiDAR as an excellent choice among fluid level measurement technologies \cite{santana_2024_development}.

Another method for monitoring water levels in open environments is the use of satellite-based remote sensing. The work \cite{jiang_2024_monitoring} employed a multi-source remote sensing data fusion method via Google Earth Engine to estimate surface water area and water level changes in nine plateau lakes in Yunnan, China, between 2003 and 2022. Their results showed high consistency with in-situ measurements, with deviations under 0.3 meters, and highlighted the influence of climate variability and human activities on lake dynamics. Similarly, \cite{ali_2024_satellite} used Sentinel-3A radar altimetry to assess the impact of dam operations on water levels in the Mekong River basin, validating satellite-derived levels with in-situ telemetry data and achieving a correlation of 0.98. Both studies confirm the viability and accuracy of satellite altimetry and image-based techniques for long-term and large-scale water level monitoring, especially in remote or transboundary regions.

\subsection{Water Level Sensors for Controlled Environments}

While the same sensors applied in rivers and lakes may still be suitable, simpler methods like resistive and capacitive sensors are commonly employed for monitoring water levels in tanks and cisterns. Resistive sensors measure electrical resistance between submerged electrodes and are inexpensive and easy to install, though they can suffer from corrosion and scaling. Capacitive sensors, in contrast, detect changes in capacitance due to varying water levels are also great options \cite{santana_2024_development, mohindru_2023_development}.

Emerging technologies show promise for these scenarios. For example, surface acoustic wave (SAW) sensors can passively detect water level changes by measuring strain or pressure variations on tank walls, converting them into frequency shifts or response signals \cite{ali_2020_saw, sreejith_2024_modeling}.

Another promising approach involves optical fiber-based sensors. These rely on the hydrostatic principle of Archimedes, where a floating element alters light transmission through a fiber to reflect changes in water level. This principle has been demonstrated in recent developments \cite{ramos_2025_high}. While highly accurate, these systems typically have limited measurement ranges and may involve more complex calibration.

% \subsection{Water Level Sensors for Controlled Environments}

% When selecting sensors for water level monitoring in controlled environments like tanks and cisterns, and where long-range sensing is not a concern, we do not need to worry as much about floating debris or extreme weather events such as floods. In these cases, the previously mentioned sensors for rivers and lakes can be used, as well as other simpler methods such as resistive water level sensors, which measure electrical resistance between two electrodes submerged in water. These sensors are inexpensive and easy to install but may be affected by corrosion and mineral deposits \cite{santana_2024_development, mohindru_2023_development}.

% For this category there are also emerging technologies with great potential. One example is passive water level sensors based on acoustic waves. The SAW sensor measures strain or pressure variations in the tank wall caused by water level changes, converting these variations into response signals, as developed by \cite{ali_2020_saw} and \cite{sreejith_2024_modeling}. Another promising technology is optical fiber-based water level sensors, which rely on the hydrostatic principle of Archimedes and use the variation of light transmitted through an optical fiber and a floating element to measure the height of the water column. This principle is demonstrated in the developments by \cite{ramos_2025_high}. These optical fiber sensors are highly accurate but have limited ranges.

\section{Sensor Systems for Detecting Chemicals and Pollutants}

\subsection{Water Quality}

Recent advancements in sensor technologies have significantly enhanced water quality monitoring. The work by \cite{ferreira_2023_conception} presents the development of a wireless sensor network (WSN) node capable of detecting pH, turbidity, and electrical conductivity (EC) sensors and classifying pollutants in aquatic environments using embedded systems, IoT, and machine learning. The system demonstrated promising results in controlled tests with seawater samples containing oil derivatives, offering low power consumption and effective pollutant classification through distributed processing and embedded neural networks. Complementing this work, \cite{nr_2025_ai} review the application of artificial intelligence (AI) in water quality assessment.

% \subsection{Soil Monitoring Review}

% Soil monitoring is an essential tool to optimize crop growth, improve production efficiency, and promote more sustainable agricultural practices. Soil sensors enable continuous measurement of physical and chemical parameters, such as moisture and nutrient concentrations, providing real-time data to support decision-making in the field. The demand for these technologies is growing, driven by population growth, the need to increase food production, and the pressure for more efficient and environmentally conscious agricultural practices. A representative example is the soil moisture sensor market, which generated around US\$147.5 million in 2020, with expectations to reach US\$360.9 million by 2027, reflecting global interest in digital agriculture technologies \cite{yin_2021_smart}.

% Historically, agricultural management recommendations were developed based on broad agroecological zones, as occurred during the Green Revolution, when the focus was solely on increasing productivity through synthetic fertilizers, without adequately considering local soil and water conditions or the associated environmental impacts. Much of this legacy still persists, with practices based on centralized procedures and generic empirical relationships between nutrients, fertilizer doses, and productivity. In this context, soil sensors emerge as a key tool to break away from this “top-down” model and enable a “bottom-up” approach, where management decisions are guided by actual data specific to each agricultural microenvironment, in both space and time \cite{viscarrarossel_2016_soil}.


\subsection{Alternative Technologies for Large-Scale Soil Monitoring}

Batteryless sensing technologies have emerged as promising alternatives for soil sensing in remote or hard to access locations. \cite{boada_2018_batteryless} introduced an NFC-based soil sensor that harvests energy from the NFC reader’s magnetic field, enabling measurements of temperature, humidity, and soil moisture without relying on batteries. The system integrates a microcontroller and stores data in NDEF format for later retrieval. 

To address the need for continuous, high-resolution monitoring of soil infiltration, \cite{sun_2024_highresolution} proposed Active Heating Optical Frequency Domain Reflectometry (AH-OFDR). This method combines improved active heating fiber optics with OFDR technology to track water infiltration continuously across soil profiles, leveraging the principle that wetter soils change temperature more efficiently. Laboratory tests validated the method against traditional distributed temperature sensing (DTS) and infrared thermography, confirming AH-OFDR’s superior spatial resolution and its potential for environmental and geotechnical soil moisture monitoring.

While LoRaWAN and traditional wireless sensor networks (WSNs) are widely adopted for monitoring in extensive agricultural fields, alternative architectures have been explored to address limitations in communication infrastructure, energy consumption, and scalability. One such approach is presented by Deng et al. \cite{deng_2020_novel}, who developed a hybrid monitoring system combining passive RFID soil sensors with mobile data collection using a patrol vehicle. This system benefits from the simplicity, low cost, and low energy requirements of RFID technology, leveraging backscatter communication to transmit environmental data such as soil conditions to a centralized platform via the mobile unit.

Another innovative method involves the use of unmanned aerial vehicles (UAVs) as mobile data collectors. Caruso et al. \cite{caruso_2021_drone} propose a solution where drones equipped with LoRa radios periodically fly over fields to collect data from distributed ground sensors. This method is particularly suitable for areas lacking fixed infrastructure or where real-time monitoring is not required. The authors present an analytical model to optimize drone path planning, sensor spacing, and flight parameters to ensure reliable data collection. Their model can be adapted to various drone types and communication protocols, offering a flexible and infrastructure-independent alternative for precision agriculture and other remote sensing applications.

% \subsection{Air and Chemical Composition Monitoring Review}

% When thinking of urban environments integrated with IoT under the concept of smart cities, one of the most relevant parameters is air quality, due to its direct impacts on public health, the environment, and the global economy. Atmospheric pollution in urban areas, with non-uniform spatial and temporal distribution, reinforces the need for monitoring systems with high spatiotemporal resolution, something traditional monitoring systems still struggle to provide at scale and with broad data coverage \cite{yi_2015_a}.

% In this context, the advancement of sensor technologies, such as MEMS and wireless sensor networks (WSNs), has driven the development of the concept of the Next Generation Air Pollution Monitoring System (TNGAPMS). For this type of application, the most concerning gases are carbon monoxide (CO), nitrogen dioxide (NO$_2$), ground-level ozone (O$_3$), and sulfur dioxide (SO$_2$). Currently, the most used and suitable sensors for monitoring these gases in urban and industrial settings are electrochemical sensors and solid-state (semiconductor) sensors, although there are also other low-cost technologies such as catalytic sensors, NDIR, and PID sensors, which are widely applied in various gas detection contexts \cite{yi_2015_a}.

\subsection{Chemical Element Detection}

The study by \cite{devkota_2017_saw} investigates SAW sensors for passive detection of gases and chemical vapors in the air, based on the interaction of the compounds with the antenna, which alters the received acoustic signal. SAW sensors are shown to be viable for detecting inorganic gases such as NH$_3$, NO$_2$, SO$_2$, CO, organic vapors such as toluene, ethanol, acetone, and even chemical warfare agents.
The study highlights the importance of material stability in extreme environments, the use of antennas for wireless operation, and low energy consumption, pointing to future research directions in sensitive materials, flexible sensors, and multi-element arrays for simultaneous gas monitoring.

\subsection{Air Pollution and Quality}

The work by \cite{karagulian_2019_review} analyzes the performance of low-cost sensors (LCS) for monitoring various air pollutants, including carbon monoxide (CO), nitrogen oxides (NO and NO$_2$), ozone (O$_3$), and particulate matter (PM$_{2.5}$). The review highlights the potential of these sensors to expand spatial coverage in both urban and remote areas and assesses different calibration methods such as MLR, ANN, SVR, and RF, considering factors like relative humidity, which significantly affects particulate measurement.

The study by \cite{yi_2015_a} presents a review of air pollution monitoring systems based on wireless sensor networks (WSNs), classifying them into three main categories: Static Sensor Networks (SSN), Community Sensor Networks (CSN), and Vehicular Sensor Networks (VSN), based on the types of sensor carriers. The analysis shows that many current solutions are already viable in terms of spatiotemporal resolution, cost, energy efficiency, ease of deployment, maintenance, and public data accessibility. However, challenges remain, such as the lack of 3D data acquisition, limitations in active monitoring capability, and the use of uncontrolled or semi-controlled carrier, factors that should be improved in the next generations of air pollution monitoring systems.

\section{Challenges and Trends} \label{cap:challenges}

\subsection{Beat Sensors}

% In this context, Wireless Sensor Networks (WSNs) have established themselves as a promising tool, offering significant advantages in terms of deployment flexibility, quick responsiveness, and cost-benefit compared to traditional monitoring infrastructures. WSNs enable dense spatial sampling and continuous data collection—critical aspects for tracking natural phenomena \cite{chen_2013_natural, pule_2017_wireless}.

% The need for real-time environmental monitoring has increased the demand for data, highlighting the importance of low-power, long-range technologies such as LoRa. These technologies are particularly suitable for applications in remote areas where communication infrastructure is limited or non-existent \cite{pule_2017_wireless, chen_2013_natural}.

Beat sensors represent an innovative approach in the field of IoT, designed specifically to minimize power consumption and cost. Introduced in 2017, these sensors operate by transmitting only ID codes wirelessly; the receiving unit (RX) decodes sensor data based on the time intervals between transmissions, referred to as “beats.” This method eliminates the need for conventional sensing circuitry and enables the use of ultra-low-power, compact, and cost-effective devices suitable for scalable deployments in smart environments. Applications include power monitoring systems such as Power Beat and DC Current Beat sensors, aimed at energy reduction in residential and commercial buildings \cite{ishibashi_2017_beat}.

Subsequent studies have demonstrated the long-term efficiency of beat sensors in energy-critical IoT scenarios. A 2019 evaluation showed that a Beat sensor powered by a CR2032 coin battery could transmit over 3.8 million ID signals, translating to more than seven years of operation, significantly outperforming conventional IoT sensors with intermittent duty cycles \cite{ishibashi_2019_long}. More recently, a 2025 study applied the beat sensing concept to water-level monitoring, proposing a solar-powered system using a \$ 72.5 BoM sensor node with LoRa communication and a quasi-isotropic $\Omega$-antenna. The device operated continuously for 21 days without a battery, achieving a 2 km communication range and consuming less than 100 $\mu$W \cite{dao_2025_lowcost}.

\subsection{Artificial Intelligence with Internet of Things( AIoT )}

The integration of Artificial Intelligence (AI) into Internet of Things (IoT) systems, referred to as AIoT, has enabled smarter, more autonomous applications across fields like smart cities, industry, and environmental monitoring. The work \cite{ghosh_2018_artificial} describe this evolution as part of a broader shift toward cyber-physical systems, where embedded devices, humans, and the environment interact to form intelligent systems. AI enhances the capability of IoT by enabling real-time analysis, predictive decision-making, and self-adaptive behavior. However, this convergence also raises challenges, including data security, ethical concerns, and the complexity of system integration.

The work of \cite{mukhopadhyay_2021_artificial} emphasize the growing use of AI-enabled sensors that are context-aware, reliable, and efficient. These sensors reduce energy use, improve response time, and support applications like predictive maintenance and smart routing. In the environmental domain, \cite{ferreira_2023_conception} apply this concept in a wireless sensor node that classifies aquatic pollutants using embedded neural networks and distributed processing. Similarly, \cite{nr_2025_ai} review the use of AI in water quality analysis, discussing its potential to improve detection accuracy and response speed while noting limitations related to data availability and transparency.

% \subsection{Security and Privacy Challenges}

% The growing ubiquity of embedded systems within IoT architectures poses significant security challenges, especially as these systems increasingly handle real-time sensing, processing, and decision-making in sensitive contexts. As shown in recent work, embedded systems now underpin critical applications that directly affect physical environments and human safety, thus making their security, reliability, and resilience essential aspects of system design and operation \cite{koulamas_2018_realtime}. Unlike traditional IT systems, embedded devices often lack robust runtime protections, and their constrained nature frequently leads to security being treated as an afterthought in the design process \cite{pimentel_2017_exploring}. The cascading effect of a single compromised device within an IoT network can jeopardize an entire system, emphasizing the need for comprehensive security strategies at both device and network levels \cite{tien_2017_internet}. Consequently, real-time embedded system architectures must prioritize integrated security mechanisms early in the design flow to reduce the attack surface, especially in edge processing scenarios where data transmission is minimized for both efficiency and security \cite{koulamas_2018_realtime}.

% Beyond software vulnerabilities, hardware-level attacks are becoming a major concern in embedded IoT environments. Recent research has revealed a class of microarchitectural attacks that exploit hardware design flaws, capable of circumventing software-level protections and undermining trust-enhancing technologies \cite{fournaris_2017_exploiting}. These low-level threats are particularly dangerous due to their stealthy nature and the lack of sufficient countermeasures from hardware vendors. Middleware, which connects heterogeneous embedded systems in complex IoT environments, can be an especially vulnerable point if underlying hardware is exposed to such attacks. The need for embedded system manufacturers to proactively provide security patches and adopt secure-by-design principles has never been more urgent. As emphasized in the literature, system architects must be aware of the evolving taxonomy of hardware attacks and incorporate defense mechanisms across the entire stack, from microarchitecture to application layer, to ensure end-to-end IoT security \cite{fournaris_2017_exploiting, pimentel_2017_exploring}.

\subsection{Security and Privacy Challenges}

The widespread integration of embedded systems in IoT architectures introduces critical security challenges, particularly as these systems perform real-time sensing and decision-making in sensitive contexts. Their constrained nature and the tendency to treat security as secondary in design make them vulnerable to attacks \cite{pimentel_2017_exploring}. As embedded systems now support safety-critical applications, ensuring reliability and resilience becomes essential \cite{koulamas_2018_realtime}. A compromised device can threaten the entire IoT network, highlighting the need for robust, integrated security—especially in edge processing, where minimizing data transfer also reduces the attack surface \cite{tien_2017_internet}.

In addition to software risks, hardware-level threats such as microarchitectural attacks are gaining attention. These stealthy exploits bypass traditional protections and compromise trust mechanisms \cite{fournaris_2017_exploiting}. Middleware connecting heterogeneous systems is particularly exposed if hardware is not adequately secured. Manufacturers must adopt secure-by-design principles and deliver timely patches, while architects must understand the full attack taxonomy to implement end-to-end protections \cite{fournaris_2017_exploiting, pimentel_2017_exploring}.

\section{Conclusion and Future Perspectives} \label{cap:conclusion}
The conclusion is at least two paragraphs long. The first paragraph should summarize the main findings of the paper, while the second paragraph should discuss the implications of these findings and suggest directions for future research.

The conclusion is at least two paragraphs long. The first paragraph should summarize the main findings of the paper, while the second paragraph should discuss the implications of these findings and suggest directions for future research.

% \section{Prepare Your Paper Before Styling}
% Before you begin to format your paper, first write and save the content as a 
% separate text file. Complete all content and organizational editing before 
% formatting. Please note sections \ref{AA}--\ref{SCM} below for more information on 
% proofreading, spelling and grammar.

% Keep your text and graphic files separate until after the text has been 
% formatted and styled. Do not number text heads---{\LaTeX} will do that 
% for you.

% \subsection{Abbreviations and Acronyms}\label{AA}
% Define abbreviations and acronyms the first time they are used in the text, 
% even after they have been defined in the abstract. Abbreviations such as 
% IEEE, SI, MKS, CGS, ac, dc, and rms do not have to be defined. Do not use 
% abbreviations in the title or heads unless they are unavoidable.

% \subsection{Units}
% \begin{itemize}
% \item Use either SI (MKS) or CGS as primary units. (SI units are encouraged.) English units may be used as secondary units (in parentheses). An exception would be the use of English units as identifiers in trade, such as ``3.5-inch disk drive''.
% \item Avoid combining SI and CGS units, such as current in amperes and magnetic field in oersteds. This often leads to confusion because equations do not balance dimensionally. If you must use mixed units, clearly state the units for each quantity that you use in an equation.
% \item Do not mix complete spellings and abbreviations of units: ``Wb/m\textsuperscript{2}'' or ``webers per square meter'', not ``webers/m\textsuperscript{2}''. Spell out units when they appear in text: ``. . . a few henries'', not ``. . . a few H''.
% \item Use a zero before decimal points: ``0.25'', not ``.25''. Use ``cm\textsuperscript{3}'', not ``cc''.)
% \end{itemize}

% \subsection{Equations}
% Number equations consecutively. To make your 
% equations more compact, you may use the solidus (~/~), the exp function, or 
% appropriate exponents. Italicize Roman symbols for quantities and variables, 
% but not Greek symbols. Use a long dash rather than a hyphen for a minus 
% sign. Punctuate equations with commas or periods when they are part of a 
% sentence, as in:
% \begin{equation}
% a+b=\gamma\label{eq}
% \end{equation}

% Be sure that the 
% symbols in your equation have been defined before or immediately following 
% the equation. Use ``\eqref{eq}'', not ``Eq.~\eqref{eq}'' or ``equation \eqref{eq}'', except at 
% the beginning of a sentence: ``Equation \eqref{eq} is . . .''

% \subsection{\LaTeX-Specific Advice}

% Please use ``soft'' (e.g., \verb|\eqref{Eq}|) cross references instead
% of ``hard'' references (e.g., \verb|(1)|). That will make it possible
% to combine sections, add equations, or change the order of figures or
% citations without having to go through the file line by line.

% Please don't use the \verb|{eqnarray}| equation environment. Use
% \verb|{align}| or \verb|{IEEEeqnarray}| instead. The \verb|{eqnarray}|
% environment leaves unsightly spaces around relation symbols.

% Please note that the \verb|{subequations}| environment in {\LaTeX}
% will increment the main equation counter even when there are no
% equation numbers displayed. If you forget that, you might write an
% article in which the equation numbers skip from (17) to (20), causing
% the copy editors to wonder if you've discovered a new method of
% counting.

% {\BibTeX} does not work by magic. It doesn't get the bibliographic
% data from thin air but from .bib files. If you use {\BibTeX} to produce a
% bibliography you must send the .bib files. 

% {\LaTeX} can't read your mind. If you assign the same label to a
% subsubsection and a table, you might find that Table I has been cross
% referenced as Table IV-B3. 

% {\LaTeX} does not have precognitive abilities. If you put a
% \verb|\label| command before the command that updates the counter it's
% supposed to be using, the label will pick up the last counter to be
% cross referenced instead. In particular, a \verb|\label| command
% should not go before the caption of a figure or a table.

% Do not use \verb|\nonumber| inside the \verb|{array}| environment. It
% will not stop equation numbers inside \verb|{array}| (there won't be
% any anyway) and it might stop a wanted equation number in the
% surrounding equation.

% \subsection{Some Common Mistakes}\label{SCM}
% \begin{itemize}
% \item The word ``data'' is plural, not singular.
% \item The subscript for the permeability of vacuum $\mu_{0}$, and other common scientific constants, is zero with subscript formatting, not a lowercase letter ``o''.
% \item In American English, commas, semicolons, periods, question and exclamation marks are located within quotation marks only when a complete thought or name is cited, such as a title or full quotation. When quotation marks are used, instead of a bold or italic typeface, to highlight a word or phrase, punctuation should appear outside of the quotation marks. A parenthetical phrase or statement at the end of a sentence is punctuated outside of the closing parenthesis (like this). (A parenthetical sentence is punctuated within the parentheses.)
% \item A graph within a graph is an ``inset'', not an ``insert''. The word alternatively is preferred to the word ``alternately'' (unless you really mean something that alternates).
% \item Do not use the word ``essentially'' to mean ``approximately'' or ``effectively''.
% \item In your paper title, if the words ``that uses'' can accurately replace the word ``using'', capitalize the ``u''; if not, keep using lower-cased.
% \item Be aware of the different meanings of the homophones ``affect'' and ``effect'', ``complement'' and ``compliment'', ``discreet'' and ``discrete'', ``principal'' and ``principle''.
% \item Do not confuse ``imply'' and ``infer''.
% \item The prefix ``non'' is not a word; it should be joined to the word it modifies, usually without a hyphen.
% \item There is no period after the ``et'' in the Latin abbreviation ``et al.''.
% \item The abbreviation ``i.e.'' means ``that is'', and the abbreviation ``e.g.'' means ``for example''.
% \end{itemize}

% \subsection{Figures and Tables}
% \paragraph{Positioning Figures and Tables} Place figures and tables at the top and 
% bottom of columns. Avoid placing them in the middle of columns. Large 
% figures and tables may span across both columns. Figure captions should be 
% below the figures; table heads should appear above the tables. Insert 
% figures and tables after they are cited in the text. Use the abbreviation 
% ``Fig.~\ref{fig}'', even at the beginning of a sentence.

% \begin{table}[htbp]
% \caption{Table Type Styles}
% \begin{center}
% \begin{tabular}{|c|c|c|c|}
% \hline
% \textbf{Table}&\multicolumn{3}{|c|}{\textbf{Table Column Head}} \\
% \cline{2-4} 
% \textbf{Head} & \textbf{\textit{Table column subhead}}& \textbf{\textit{Subhead}}& \textbf{\textit{Subhead}} \\
% \hline
% copy& More table copy$^{\mathrm{a}}$& &  \\
% \hline
% \multicolumn{4}{l}{$^{\mathrm{a}}$Sample of a Table footnote.}
% \end{tabular}
% \label{tab1}
% \end{center}
% \end{table}

% \begin{figure}[htbp]
% \centerline{\includegraphics{fig1.png}}
% \caption{Example of a figure caption.}
% \label{fig}
% \end{figure}

% Figure Labels: Use 8 point Times New Roman for Figure labels. Use words 
% rather than symbols or abbreviations when writing Figure axis labels to 
% avoid confusing the reader. As an example, write the quantity 
% ``Magnetization'', or ``Magnetization, M'', not just ``M''. If including 
% units in the label, present them within parentheses. Do not label axes only 
% with units. In the example, write ``Magnetization (A/m)'' or ``Magnetization 
% \{A[m(1)]\}'', not just ``A/m''. Do not label axes with a ratio of 
% quantities and units. For example, write ``Temperature (K)'', not 
% ``Temperature/K''.

\section*{Acknowledgment}

The preferred spelling of the word ``acknowledgment'' in America is without 
an ``e'' after the ``g''. Avoid the stilted expression ``one of us (R. B. 
G.) thanks $\ldots$''. Instead, try ``R. B. G. thanks$\ldots$''. Put sponsor 
acknowledgments in the unnumbered footnote on the first page.

\begin{thebibliography}{00}
    
\bibitem{javaid_2021_sensors} M. Javaid, A. Haleem, S. Rab, R. P. Singh, and R. Suman, ``Sensors for daily life: A review,'' \emph{Sensors International}, vol. 2, p. 100121, 2021. doi: 10.1016/j.sintl.2021.100121.
\bibitem{yi_2015_a} W. Yi, K. Lo, T. Mak, K. Leung, Y. Leung, and M. Meng, ``A survey of wireless sensor network based air pollution monitoring systems,'' \emph{Sensors}, vol. 15, pp. 31392--31427, Dec. 2015. doi: 10.3390
\bibitem{mohindru_2023_development} P. Mohindru, ``Development of liquid level measurement technology: A review,'' \emph{Flow Measurement and Instrumentation}, vol. 89, p. 102295, Mar. 2023. doi: 10.1016/j.flowmeasinst.2022.102295.
\bibitem{wu_2023_a} Z. Wu, Y. Huang, K. Huang, K. Yan, and H. Chen, ``A Review of Non-Contact Water Level Measurement Based on Computer Vision and Radar Technology,'' \emph{Water}, vol. 15, p. 3233, Jan. 2023. doi: 10.3390/w15183233.
\bibitem{santana_2024_development} V. Santana, R. E. Salustiano, and R. Tiezzi, ``Development and Calibration of a Low-Cost LIDAR Sensor for Water Level Measurements,'' \emph{Flow Measurement and Instrumentation}, vol. 2024, pp. 102729--102729, Oct. 2024. doi: 10.1016/j.flowmeasinst.2024.102729.
\bibitem{li_2022_a} N. Li, C. P. Ho, J. Xue, L. W. Lim, G. Chen, Y. H. Fu, and L. Y. T. Lee, 
``A Progress Review on Solid-State LiDAR and Nanophotonics-Based LiDAR Sensors,'' 
Laser and Photonics Reviews, vol. 16, p. 2100511, Aug. 2022. doi: 10.1002/lpor.202100511.
\bibitem{mohammadrezamasoudimoghaddam_2024_a} M. MasoudiMoghaddam, J. Yazdi, and M. Shahsavandi, 
``A Low-Cost Ultrasonic Sensor for Online Monitoring of Water Levels in Rivers and Channels,'' 
Flow Measurement and Instrumentation, vol. 102, p. 102777, Dec. 2024. Elsevier BV. doi: 10.1016/j.flowmeasinst.2024.102777.
\bibitem{paul_2020_a} J. D. Paul, W. Buytaert, and N. Sah, 
``A Technical Evaluation of Lidar-Based Measurement of River Water Levels,'' 
Water Resources Research, vol. 56, Apr. 2020. doi: 10.1029/2019wr026810.
\bibitem{yukawa_2025_an} C. Yukawa, T. Oda, T. Sato, M. Hirota, K. Katayama, and L. Barolli, 
``An Intelligent Water Level Estimation System Considering Water Level Device Gauge Image Recognition and Wireless Sensor Networks,'' 
Journal of Sensor and Actuator Networks, vol. 14, p. 13, Jan. 2025. doi: 10.3390/jsan14010013.
\bibitem{ferreira_2023_conception} Y. Ferreira, C. Silvério, and J. Viana, 
``Conception and Design of WSN Sensor Nodes Based on Machine Learning, Embedded Systems and IoT Approaches for Pollutant Detection in Aquatic Environments,'' 
IEEE Access, vol. 11, pp. 117040--117052, Jan. 2023. doi: 10.1109/access.2023.3325760.
\bibitem{viscarrarossel_2016_soil} R. A. Viscarra Rossel and J. Bouma, 
``Soil sensing: A new paradigm for agriculture,'' 
Agricultural Systems, vol. 148, pp. 71--74, Oct. 2016. doi: 10.1016/j.agsy.2016.07.001.
\bibitem{yin_2021_smart} H. Yin, Y. Cao, B. Marelli, X. Zeng, A. J. Mason, and C. Cao, 
``Smart Agriculture Systems: Soil Sensors and Plant Wearables for Smart and Precision Agriculture (Adv. Mater. 20/2021),'' 
Advanced Materials, vol. 33, p. 2170156, May 2021. doi: 10.1002/adma.202170156.
\bibitem{deng_2020_novel} F. Deng, P. Zuo, K. Wen, and X. Wu, 
``Novel soil environment monitoring system based on RFID sensor and LoRa,'' 
Computers and Electronics in Agriculture, vol. 169, p. 105169, Feb. 2020. doi: 10.1016/j.compag.2019.105169.
\bibitem{akhileshnagpure_2022_water} A. Nagpure, G. Reddy, and S. Naidu, 
``Water Level Controller Using Ultrasonic Sensor,'' 
International Journal for Research in Applied Science and Engineering Technology, vol. 10, pp. 242--257, Dec. 2022. 
doi: 10.22214/ijraset.2022.47856.
\bibitem{boada_2018_batteryless} M. Boada, A. Lazaro, R. Villarino, and D. Girbau, 
``Battery-Less Soil Moisture Measurement System Based on a NFC Device With Energy Harvesting Capability,'' 
IEEE Sensors Journal, vol. 18, pp. 5541--5549, Jul. 2018. 
doi: 10.1109/jsen.2018.2837388.
\bibitem{devkota_2017_saw} J. Devkota, P. Ohodnicki, and D. Greve, 
``SAW Sensors for Chemical Vapors and Gases,'' 
Sensors, vol. 17, p. 801, Apr. 2017. 
doi: 10.3390/s17040801.
\bibitem{karagulian_2019_review} F. Karagulian, M. Barbiere, A. Kotsev, L. Spinelle, M. Gerboles, F. Lagler, N. Redon, S. Crunaire, and A. Borowiak, 
``Review of the Performance of Low-Cost Sensors for Air Quality Monitoring,'' 
Atmosphere, vol. 10, p. 506, Aug. 2019. doi: 10.3390/atmos10090506.
\bibitem{jiang_2024_monitoring} Z. Jiang and L. Hong, 
``Monitoring of Surface Water Area and Water Level Changes in Nine Plateau Lakes in Yunnan and Analysis of Influencing Factors,'' 
in *Proc. 2024 IEEE 6th Advanced Information Management, Communicates, Electronic and Automation Control Conf. (IMCEC)*, 
pp. 1027--1031, May 2024. doi: 10.1109/imcec59810.2024.10575197.
\bibitem{ali_2024_satellite} T. Ali, A. Zaidi, J. Rehman, F. Noor, F. Naz, and S. Jamali, 
``Satellite Radar Altimetry Insights into Dam-Induced Changes and Accuracy of Water Level Estimation for the Mekong River,'' 
in *Proc. IGARSS 2022 - IEEE Int. Geosci. Remote Sens. Symp.*, vol. 570, pp. 5063--5066, Jul. 2024. 
doi: 10.1109/igarss53475.2024.10642418.
\bibitem{tamari_2016_flash} S. Tamari and V. Guerrero-Meza, 
``Flash Flood Monitoring with an Inclined Lidar Installed at a River Bank: Proof of Concept,'' 
Remote Sensing, vol. 8, p. 834, Oct. 2016. doi: 10.3390/rs8100834.
\bibitem{pereira_2022_evaluation} T. S. R. Pereira, T. P. de Carvalho, T. A. Mendes, and K. T. M. Formiga, 
``Evaluation of Water Level in Flowing Channels Using Ultrasonic Sensors,'' 
Sustainability, vol. 14, p. 5512, May 2022. doi: 10.3390/su14095512.
\bibitem{bresnahan_2023_a} P. Bresnahan, E. Briggs, B. Davis, A. Rodriguez, L. Edwards, C. Peach, N. Renner, H. Helling, and M. Merrifield, 
``A Low-Cost, DIY Ultrasonic Water Level Sensor for Education, Citizen Science, and Research,'' 
Oceanography, vol. 36, 2023. doi: 10.5670/oceanog.2023.101.
\bibitem{behroozpour_2017_lidar} B. Behroozpour, P. A. M. Sandborn, M. C. Wu, and B. E. Boser, 
``Lidar System Architectures and Circuits,'' 
IEEE Communications Magazine, vol. 55, pp. 135--142, Oct. 2017. doi: 10.1109/mcom.2017.1700030.
\bibitem{fernandezdiaz_2014_early} J. C. Fernandez-Diaz, C. L. Glennie, W. E. Carter, R. L. Shrestha, M. P. Sartori, A. Singhania, C. J. Legleiter, and B. T. Overstreet, 
``Early Results of Simultaneous Terrain and Shallow Water Bathymetry Mapping Using a Single-Wavelength Airborne LiDAR Sensor,'' 
IEEE Journal of Selected Topics in Applied Earth Observations and Remote Sensing, vol. 7, pp. 623--635, Feb. 2014. 
doi: 10.1109/jstars.2013.2265255.
\bibitem{smart_2009_river} G. Smart, J. Bind, and M. Duncan, 
``River bathymetry from conventional LiDAR using water surface returns,'' 
in *Proc. 18th World IMACS / MODSIM Congress*, 2009. [Online]. Available: https://mssanz.org.au/modsim09/F13/smart.pdf
\bibitem{ali_2020_saw} S. F. Ali, N. Mandal, P. Maurya, and A. Lata, 
``SAW Sensor Based a Novel Hydrostatic Liquid Level Measurement,'' 
in *Proc. IECON 2020 - 46th Annual Conf. IEEE Industrial Electronics Society*, pp. 724--729, Oct. 2020. 
doi: 10.1109/iecon43393.2020.9254540.
\bibitem{sreejith_2024_modeling} V. S. Sreejith and H. Zhang, 
``Modeling and Testing of a Highly Sensitive Surface Acoustic Wave Pressure Sensor for Liquid Depth Measurements,'' 
Sensors and Actuators A: Physical, vol. 372, p. 115377, Jul. 2024. doi: 10.1016/j.sna.2024.115377.
\bibitem{ramos_2025_high} C. C. Ramos, J. Preizal, X. Hu, C. Caucheteur, G. Woyessa, O. Bang, A. M. Rocha, and R. Oliveira, 
``High Resolution Liquid Level Sensor Based on Archimedes’ Law of Buoyancy Using Polymer Optical Fiber Bragg Gratings,'' 
Measurement, vol. 252, p. 117368, Mar. 2025. Elsevier BV. doi: 10.1016/j.measurement.2025.117368.
\bibitem{jonkman_2005_global} S. N. Jonkman, 
``Global Perspectives on Loss of Human Life Caused by Floods,'' 
Natural Hazards, vol. 34, pp. 151--175, Feb. 2005. doi: 10.1007/s11069-004-8891-3.
\bibitem{borga_2014_hydrogeomorphic} M. Borga, M. Stoffel, L. Marchi, F. Marra, and M. Jakob, 
``Hydrogeomorphic Response to Extreme Rainfall in Headwater Systems: Flash Floods and Debris Flows,'' 
Journal of Hydrology, vol. 518, pp. 194--205, Oct. 2014. doi: 10.1016/j.jhydrol.2014.05.022.
\bibitem{hall_2014_understanding} J. Hall \textit{et al.}, 
``Understanding Flood Regime Changes in Europe: A State-of-the-Art Assessment,'' 
Hydrology and Earth System Sciences, vol. 18, pp. 2735--2772, Jul. 2014. doi: 10.5194/hess-18-2735-2014.
\bibitem{bragana_2024_anlise} C. G. J. F. N. Bragança, 
``Análise da Influência dos Fenômenos El Niño e La Niña na Ocorrência de Eventos Climáticos de Seca e Enchente,'' 
Dissertação de Mestrado, 2024.
\bibitem{iqbal_2021_how} U. Iqbal, P. Perez, W. Li, and J. Barthelemy, 
``How Computer Vision Can Facilitate Flood Management: A Systematic Review,'' 
International Journal of Disaster Risk Reduction, vol. 53, p. 102030, Feb. 2021. doi: 10.1016/j.ijdrr.2020.102030.
\bibitem{lin_2020_semantic} F. Lin, Z. Yu, Q. Jin, and A. You, 
``Semantic Segmentation and Scale Recognition–Based Water-Level Monitoring Algorithm,'' 
Journal of Coastal Research, vol. 105, 2020. doi: 10.2112/jcr-si105-039.1.
\bibitem{lo_2015_visual} S.-W. Lo, J.-H. Wu, F.-P. Lin, and C.-H. Hsu, 
``Visual Sensing for Urban Flood Monitoring,'' 
Sensors, vol. 15, pp. 20006--20029, Aug. 2015. doi: 10.3390/s150820006.
\bibitem{chen_2013_natural} D. Chen, Z. Liu, L. Wang, M. Dou, J. Chen, and H. Li, 
``Natural Disaster Monitoring with Wireless Sensor Networks: A Case Study of Data-Intensive Applications upon Low-Cost Scalable Systems,'' 
Mobile Networks and Applications, vol. 18, pp. 651--663, Aug. 2013. doi: 10.1007/s11036-013-0456-9.

\bibitem{yellampalli_2021_wireless} S. Yellampalli, Ed., 
*Wireless Sensor Networks - Design, Deployment and Applications*, IntechOpen, Sep. 2021. doi: 10.5772/intechopen.77917. [Online]. Available: https://www.intechopen.com/books/8086

\bibitem{pule_2017_wireless} M. Pule, A. Yahya, and J. Chuma, 
``Wireless Sensor Networks: A Survey on Monitoring Water Quality,'' 
Journal of Applied Research and Technology, vol. 15, pp. 562--570, Dec. 2017. doi: 10.1016/j.jart.2017.07.004.

\bibitem{nr_2025_ai} W. B. N.R, S. Palarimath, H. Gunasekaran, S. W. Haidar, S. G. S. Subitha, and J. Sarmila, 
``AI Modeling and Water Quality Sensing Technique Proffers Water Security: An Open Review,'' 
in *Proc. 2022 Int. Conf. Electronics and Renewable Systems (ICEARS)*, pp. 1028--1034, Feb. 2025. doi: 10.1109/icears64219.2025.10940077.

\bibitem{mukhopadhyay_2021_artificial} S. C. Mukhopadhyay, S. K. S. Tyagi, N. K. Suryadevara, V. Piuri, F. Scotti, and S. Zeadally, 
``Artificial Intelligence-Based Sensors for Next Generation IoT Applications: A Review,'' 
IEEE Sensors Journal, vol. 21, pp. 1--1, 2021. doi: 10.1109/jsen.2021.3055618.

\bibitem{teng_2014_soil} Y. Teng, J. Wu, S. Lu, Y. Wang, X. Jiao, and L. Song, 
``Soil and Soil Environmental Quality Monitoring in China: A Review,'' 
Environment International, vol. 69, pp. 177--199, Aug. 2014. doi: 10.1016/j.envint.2014.04.014.

\bibitem{queiroz_2020_sensors} D. M. de Queiroz, A. L. de F. Coelho, D. S. M. Valente, and J. K. Schueller, 
``Sensors Applied to Digital Agriculture: A Review,'' 
Revista Ciência Agronômica, vol. 51, 2020. doi: 10.5935/1806-6690.20200086.

\bibitem{tien_2017_internet} J. M. Tien, 
``Internet of Things, Real-Time Decision Making, and Artificial Intelligence,'' 
Annals of Data Science, vol. 4, pp. 149--178, May 2017.

\bibitem{haroun_2021_progress} A. Haroun, X. Le, S. Gao, B. Dong, T. He, Z. Zhang, F. Wen, S. Xu, and C. Lee, 
``Progress in Micro/Nano Sensors and Nanoenergy for Future AIoT-Based Smart Home Applications,'' 
Nano Express, vol. 2, p. 022005, Apr. 2021. doi: 10.1088/2632-959x/abf3d4.

\bibitem{ghosh_2018_artificial} A. Ghosh, D. Chakraborty, and A. Law, 
'Artificial Intelligence in Internet of Things,' 
CAAI Transactions on Intelligence Technology, vol. 3, pp. 208--218, Dec. 2018. doi: 10.1049/trit.2018.1008. [Online]. Available: https://ietresearch.onlinelibrary.wiley.com/doi/10.1049/trit.2018.1008

\bibitem{caruso_2021_drone} A. Caruso, S. Chessa, S. Escolar, J. Barba, and J. C. Lopez, “Collection of Data with Drones in Precision Agriculture: Analytical Model and LoRa Case Study,” IEEE Internet of Things Journal, vol. 2021, pp. 1–1, 2021, doi: 10.1109/JIOT.2021.3075561.

\bibitem{sun_2024_highresolution} C. Sun, C.-S. Tang, F. Vahedifard, Q. Cheng, A. Dong, T.-F. Gao, and B. Shi, "High-resolution monitoring of soil infiltration using distributed fiber optic," J. Hydrol., vol. 640, p. 131691, Jul. 2024, doi: 10.1016/j.jhydrol.2024.131691.
 
\bibitem{dao_2025_lowcost} 
M.-H. Dao, K. Ishibashi, T.-A. Nguyen, D.-H. Bui, H. Hirayma, T.-A. Tran, and X.-T. Tran, 
"Low-cost, high accuracy, and long communication range energy-harvesting Beat Sensor with LoRa and $\Omega$-antenna for water-level monitoring,"
*IEEE Sensors Journal*, vol. 25, pp. 1--1, Jan. 2025. doi: 10.1109/jsen.2025.3533014.

\bibitem{ishibashi_2017_beat} 
K. Ishibashi, R. Takitoge, and S. Ishigaki, 
"Beat sensors IoT technology suitable for energy saving," 
in *Proc. 2017 7th Int. Conf. on Integrated Circuits, Design, and Verification (ICDV)*, pp. 52--55, Oct. 2017. doi: 10.1109/icdv.2017.8188637.

\bibitem{ishibashi_2019_long} 
K. Ishibashi, R. Takitoge, D. Manyvone, N. Ono, and S. Yamaguchi, 
"Long battery life IoT sensing by Beat Sensors," 
in *Proc. 2019 IEEE Int. Conf. on Industrial Cyber Physical Systems (ICPS)*, pp. 430--435, May 2019. doi: 10.1109/icphys.2019.8780159.

\bibitem{pimentel_2017_exploring}
Pimentel, A. D. (2017). Exploring exploration: A tutorial introduction to embedded systems design space exploration. \textit{IEEE Design \& Test}, 1, 77--90. doi:10.1109/MDAT.2016.2626445.

\bibitem{koulamas_2018_realtime}
Koulamas, C., \& Lazarescu, M. (2018). Real-time embedded systems: Present and future. Electronics, 9, 205. doi:10.3390/electronics7090205.

\bibitem{fournaris_2017_exploiting}
Fournaris, A., Pocero Fraile, L., \& Koufopavlou, O. (2017). Exploiting hardware vulnerabilities to attack embedded system devices: A survey of potent microarchitectural attacks. Electronics,3, 52. doi:10.3390/electronics6030052.

\end{thebibliography}
 \vspace{12pt}
 \color{red}

IS REF VALID? ------  C. G. J. F. N. Braganc¸a, “An´alise da Influˆencia dos Fenˆomenos El Ni˜no
e La Ni˜na na Ocorrˆencia de Eventos Clim´aticos de Seca e Enchente,”
Dissertac¸ ˜ao de Mestrado, 2024.  
% IEEE conference templates contain guidance text for composing and formatting conference papers. Please ensure that all template text is removed from your conference paper prior to submission to the conference. Failure to remove the template text from your paper may result in your paper not being published.



\end{document}
